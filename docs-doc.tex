%! Author = Jander Moreira
%! Date = 2024

\documentclass[11pt, outputdir = ./out]{article}
\usepackage[T1]{fontenc}

\usepackage{docs}

\title{The \PackageName{docs} package\footnote{This document is refers to the package version \DocsVersion, dated \DocsDate. Compilation date: \today.}}
\author{Jander Moreira -- \texttt{moreira.jander@gmail.com}}
\date{October 27, 2024}


\begin{document}
\maketitle
\tableofcontents

\DocsCreate{Option}[
    prefix = \textbackslash,
    arguments prefix = \texttt{: },
    index entry = Options,
    index remark = { \textsuperscript{\textit{opt.}}},
    font = \scshape\bfseries,
    color = blue,
]

\section{Introduction}
\index{intro}


\section{Package usage}
The package \PackageName{docs}.

\DocsCreate{Package}[color = blue, prefix = \textbackslash]
\begin{Package*}{usepackage}{\OArg{package options}\FArg{docs}}{}
    No \Argument{package options} for now.
\end{Package*}

\section{References and index support}

An \textit{element} in the scope of this document refers to an item that can be highlighted and referenced, such as macros, options and environments, for example.

\subsection{Creating elements}

\section{Code examples}

Minted \texttt{outputdir}.

\section{Indexing elements}

\section{Version control}

\printindex

\end{document}